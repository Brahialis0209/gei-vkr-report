\chapter{Описание используемых методов} \label{ch2}
	
% не рекомендуется использовать отдельную section <<введение>> после лета 2020 года
%\section{Введение} \label{ch2:intro}


\section{Методы обработки изображений реализованные в пакете ProStack} \label{ch2:title-abbr} %название по-русски
В данной работе производится модификация и улучшение методов обработки изображений реализованных в пакете ProStack.

В пакете ProStack реализованы стандартные и проблемно-ориентированные методы
обработки изображений а также методы для получения количественных данных из изображений, полученных на световом или конфокальном микроскопе. Пакет имеет графический
интерфейс, для построения сложных сценариев. 

Механизм обработки изображения в данном пакете представляет из себя
своего рода конвеер - множество зависимых
друг от друга различных операций, записанных в один сценарий.

Все методы в рамках пакета разделены на десять классов.
\begin{itemize}
	\item Комбинирование (Получение одного изображения из нескольких входов)
	\item Выделение объектов
	\item Корректировка (Повышение качества изображения)
	\item Сегментация (Разделение изображения на части/зоны)
	\item Восстановление
	\item Морфология (Морфологические операции)
	\item Геометрия (Изменение свойст изображений)
	\item Преобразование
	\item Арифметика (Алгебраические операции)
	\item Разное
\end{itemize}

Рассмотрим некоторые морфологические операции, которые реализованы в пакете, а также применялись для извлечение количественных данных из изображений мозга мушки.

Для улучшения сегментации(например обработки фона) применяют операцию морфологического размыкания - комбинацию операций эрозии и наращивания. Расссмотрим операции на простом примере. Пусть у нас есть следующее бинарное изображение и структурный элемент:
\begin{figure}[H]
	\minipage{0.48\textwidth}
	\includegraphics[width=5cm, height=5cm]{morf_bin_example}
	\caption{Бинарное изображение B}\label{morf_bin_example}
	\endminipage\hfill
	\minipage{0.48\textwidth}
	\includegraphics[width=5cm, height=5cm]{morf_struct_example}
	\caption{Структурный элемент S}\label{morf_struct_example}
	\endminipage\hfill
\end{figure}

	\textbf{Наращивание} — Структурный элемент S применяется ко всем пикселам бинарного изображения. Каждый раз, когда начало координат структурного элемента совмещается с единичным бинарным пикселом, ко всему структурному элементу применяется перенос и последующее логическое сложение с соответствующими пикселами бинарного изображения. Результаты логического сложения записываются в выходное бинарное изображение, которое изначально инициализируется пустыми значениями.
	
	\begin{figure}[H]
		\centering
		\includegraphics[width=5cm, height=5cm]{morf_narach}
		\caption{Наращивание бинарного изображения структурным элемнтом}
		\label{morf_narach}
	\end{figure}
	
	\textbf{Эрозия} — При выполнении операции эрозии структурный элемент тоже проходит по всем пикселам изображения. Если в некоторой позиции каждый единичный пиксел структурного элемента совпадет с единичным пикселом бинарного изображения, то выполняется логическое сложение центрального пиксела структурного элемента с соответствующим пикселом выходного изображения.
	
	\begin{figure}[H]
		\centering
		\includegraphics[width=5cm, height=5cm]{morf_eroz}
		\caption{Эрозия бинарного изображения структурным элемнтом}
		\label{morf_eroz}
	\end{figure}
	В результате применения операции эрозии все объекты, меньшие чем структурный элемент, стираются, объекты, соединённые тонкими линиями становятся разъединёнными и размеры всех объектов уменьшаются.
	
	\textbf{Размыкание} — Операция эрозии полезна для удаления малых объектов и различных шумов, но у этой операции есть недостаток – все остающиеся объекты уменьшаются в размере. Этого эффекта можно избежать, если после операции эрозии применить операцию наращивания с тем же структурным элементом.
	Размыкание отсеивает все объекты, меньшие чем структурный элемент, но при этом помогает избежать сильного уменьшения размера объектов. Также размыкание идеально подходит для удаления линий, толщина которых меньше, чем диаметр структурного элемента. Также важно помнить, что после этой операции контуры объектов становятся более гладкими.
	\begin{figure}[H]
		\centering
		\includegraphics[width=5cm, height=5cm]{morf_razmik}
		\caption{Размыкание бинарного изображения структурным элемнтом}
		\label{morf_razmik}
	\end{figure}

Далеее рассмотрим довольно распространенную операцию в обработе изображений - \textit{выделение границ}. Данная операция, а именно оператор Кэнни был применен в этой работе для выделения комплексов молекул РНК.

Границы объектов на изображении в значительной степени уменьшают количество данных, которые необходимо обработать, и в то же время сохраняет важную информацию об объектах на изображении, их форму, размер, количество. Главной особенностью техники обнаружения границ является возможность извлечь точную линию с хорошей ориентацией.

Граница является местным изменением яркости на изображении. Они, как правило, проходят по краю между двумя областями. С помощью границ можно получить базовые знания об изображении. Функции их получения используются передовыми алгоритмами компьютерного зрения и таких областях, как медицинская обработка изображений, биометрия и тому подобные. Обнаружение границ — активная область исследований, так как он облегчает высокоуровневый анализ изображений. На полутоновых изображениях существует три вида разрывов: точка, линия и граница. Для обнаружения всех трех видов неоднородностей могут быть использованы пространственные маски.

В технической литературе приведено и описано большое количество алгоритмов выделения контуров и границ. В данной работе рассмотрены наиболее популярные методы. К ним относятся: оператор Робертса, Собеля, Превитта,алгоритм Кэнни.
\begin{center}
	\textbf{Фильтрация}
\end{center}
Все следующие методы основываются на одном из базовых свойств сигнала яркости – \textbf{разрывности} (резких изменений значений интенсивности). Наиболее общим способом поиска \textbf{разрывов} является обработка изображения с помощью скользящей маски, называемой также фильтром, ядром, окном или шаблоном, которая представляет собой некую квадратную матрицу, соответствующую указанной группе пикселей исходного изображения. Элементы матрицы принято называть коэффициентами. Оперирование такой матрицей в каких-либо локальных преобразованиях называется \textbf{фильтрацией} или \textbf{пространственной фильтрацией}.    
	Схема пространственной фильтрации иллюстрируется на рисунке \ref{mask}:
	\begin{figure}[H]
		\centering
		\includegraphics[width=14cm, height=12cm]{mask}
		\caption{Схема пространственной фильтрации}
		\label{mask}
\end{figure}

Процесс основан на простом перемещении маски фильтра от точки к точке изображения; в каждой точке \textbf{(x,y)} отклик фильтра вычисляется с использованием предварительно заданных связей. В случае линейной пространственной фильтрации отклик задается суммой произведения коэффициентов фильтра на соответствующие значения пикселей в области, покрытой маской фильтра. Для маски 3х3 элемента, показанной на рисунке \ref{mask}, результат (отклик) \textbf{R} линейной фильтрации в точке \textbf{(x,y)} изображения составит:
$R = w(-1, -1) * f(x - 1, y - 1) + w(-1, 0) * f(x - 1, y) + ...w(0,0) * f(x, y) + ... + w(1, 0) * f(x + 1, y) + w(1,1) * f(x + 1, y + 1)$ \\
что, как видно, есть сумма произведений коэффициентов маски на значения пикселей непосредственно под маской. В частности заметим, что коэффициент \textbf{w(0,0)} стоит при значении \textbf{f(x,y)}, указывая тем самым, что маска центрирована в точке \textbf{(x,y)}.\\
При обнаружении перепадов яркости используются дискретные аналоги производных первого и второго порядка.Первая производная одномерной функции f(x) определяется как разность значений соседних элементов: \\ $\frac{df}{dx} = f(x + 1) - f(x)$.\\
	Аналогично, вторая производная определяется как разность соседних значений первой производной:\\
	$\frac{df^2}{dx^2} = f(x + 1) - f(x - 1) - 2f(x)$.\\
	По определению, градиент изображения f(x,y) в точке (x,y) — это вектор:\\
	$\nabla f = [\frac{G_x}{G_y}] = \dfrac{\frac{df}{dx}}{\frac{df}{dy}}$.\\
	Направление вектора градиента совпадает с направлением максимальной скорости изменения функции f в точке (x,y).
	Важную роль при обнаружении контуров играет модуль этого вектора, который обозначается $ |\nabla f| $ и  равен\\
	$|\nabla f| = \sqrt{G_x^2 + G_y^2}$. Эта величина равна значению максимальной скорости изменения функции f в точке (x,y). \cite{find-contours-habr}
\textbf{\begin{center}
		Оператор Робертса
\end{center}}
Пусть область 3х3, показанная на рисунке ниже (см. \ref{Robert_1}), представляет собой значения яркости в окрестности некоторого элемента изображения.
	\begin{figure}[H]
		\centering
		\includegraphics[width=5cm, height=3cm]{Robert_1}
		\caption{Окрестность 3х3 внутри изображения}
		\label{Robert_1}
	\end{figure}  
	Один из простейших способов нахождения первых частных производных в точке $z_5$ состоит в применении следующего перекрестного градиентного оператора Робертса: $G_x = z_9 - z_5$ и $G_y = z_8 - z_6$\\
	Эти производные могут быть реализованы путем обработки всего изображения с помощью оператора, описываемого масками на рисунке \ref{Robert_2}, используя процедуру фильтрации, описанную ранее.
	\begin{figure}[H]
		\centering
		\includegraphics[width=4cm, height=2cm]{Robert_2}
		\caption{Маски оператора Робертса}
		\label{Robert_2}
	\end{figure}
	Реализация масок размерами 2х2 не очень удобна, т.к. у них нет четко выраженного центрального элемента, что существенно отражается на результате выполнения фильтрации. Но этот «минус» порождает очень полезное свойство данного алгоритма – высокую скорость обработки изображения.
\textbf{\begin{center}
		Оператор Превитта
\end{center}}
Оператор Превитта, так же как и оператор Робертса, оперирует с областью изображения 3х3, представленной на рисунке \ref{Robert_1}, только использование такой маски задается другими выражениями:\\
	$G_x = (z_7 + z_8 + z_9) - (z_1 + z_2 + z_3)$ и $G_y = (z_3 + z_6 + z_9) - (z_1 + z_4 + z_7)$\\
	В этих формулах разность между суммами по верхней и нижней строкам окрестности 3х3 является приближенным значением производной по оси x, а разность между суммами по первому и последнему столбцам этой окрестности – производной по оси y. Для реализации этих формул используется оператор, описываемый масками на рисунке \ref{Previt_1}, который называется оператором Превитта.  
	\begin{figure}[H]
		\centering
		\includegraphics[width=8cm, height=3cm]{Previt_1}
		\caption{Маски оператора Превитта}
		\label{Previt_1}
\end{figure}
\textbf{\begin{center}
		Оператор Собеля
\end{center}}
Оператор Собеля тоже использует область изображения 3х3, отображенную на рисунке \ref{Robert_1}. Он довольно похож на оператор Превитта, а видоизменение заключается в использовании весового коэффициента 2 для средних элементов:
	$G_x = (z_7 + 2z_8 + z_9) - (z_1 + 2z_2 + z_3)$ и $G_y = (z_3 + 2z_6 + z_9) - (z_1 + 2z_4 + z_7)$\\
	Это увеличенное значение используется для уменьшения эффекта сглаживания за счет придания большего веса средним точкам.\\
	Маски, используемые оператором Собеля, отображены на рисунке ниже (см. рис. \ref{Sobol_1}).
\begin{figure}[H]
	\centering
	\includegraphics[width=8cm, height=3cm]{Sobol_1}
	\caption{Маски оператора Собеля}
	\label{Sobol_1}
\end{figure}

Рассмотренные выше маски применяются для получения составляющих градиента $G_x$ и $G_y$ . Для вычисления величины градиента эти составляющие необходимо использовать совместно:\\
$|\nabla f| = \sqrt{G_x^2 + G_y^2}$\\
Еще один алгоритм который является модификацией вышеперечисленных а также применялся в данной работе:
\textbf{\begin{center}
		Детектор границ Канни
\end{center}}
Детектор границ Канни является одной из самых популярных алгоритмов обнаружения контуров. Впервые он был предложен Джоном Канни в магистерской диссертации в 1983 году, и до сих пор является лучше многих алгоритмов, разработанных позднее. Важным шагом в данном алгоритме является устранение шума на контурах, который в значительной мере может повлиять на результат, при этом необходимо максимально сохранить границы. Для этого необходим тщательный подбор порогового значения при обработке данным методом.
 
	Алгоритм: 
	\begin{itemize}
		\item Размытие исходного изображения
		\item Выполнить поиск градиента. Границы намечаются там, где градиент принимает максимальное значение
		\item Подавление не-максимумов. Только локальные максимумы отмечаются как границы
		\item Итоговые границы определяются путем подавления всех краев, не связанных с определенными границами.
	\end{itemize}

	В отличии от операторов Робертса и Собеля, алгоритм Канни не очень восприимчив к шуму на изображении.

%%%%
%%		
%%  \input{...} commands are used only to sychronize some parts of the text with the author guide. Authors are free to type the text directly in .tex-files   
%%  \input{...} комманды используются только, чтобы синхронизировать части текта с рекомендациями авторам. Авторы  вольны вносить текст непосредственно в файл главы  
%%  


	


	
\section{Название параграфа} \label{ch2:sec-abbr} %название по-русски
	
Название параграфа оформляется с помощью команды \verb|\section{...}|, название главы --- \verb|\chapter{...}|. 
	

\subsection{Название подпараграфа} \label{ch2:subsec-title-abbr} %название по-русски


Название подпараграфа оформляется с помощью команды  \texttt{\textbackslash{}subsection\{...\}}.


%\subsubsection{Название подподпараграфа} \label{ch2:subsubsec-title-abbr} %название по-русски
	
Использование подподпараграфов в основной части крайне не рекомендуется. В случае использования, необходимо вынести данный номер в содержание.	
Название подпараграфа оформляется с помощью команды  \texttt{\textbackslash{}subsubsecti\-on\{...\}}.



\input{my_folder/tex/enumeration} % правила использования перечислений	

	
Оформление псевдокода необходимо осуществлять с помощью пакета \verb|algorithm2e| в окружении \verb|algorithm|. Данное окружение интерпретируется в шаблоне как рисунок. Пример оформления псевдокода алгоритма приведён на \firef{alg:AlgoFDSCALING}. 
	
	
\input{my_folder/tex/pseudocode-agl-DTestsFDScaling} % пример оформления псевдокода алгоритма 	

	
\section{Название параграфа} \label{ch2:sec-very-short-title} %название по-русски


	
\input{my_folder/tex/eq-equation-multilined} % пример оформления одиночной формулы в несколько строк

\input{my_folder/tex/fig-spbpu-sc-four-in-one} % пример подключения 4х иллюстраций в одном рисунке

%\input{my_folder/tex/fig-spbpu-whitehall-three-in-one} % пример подключения 3х иллюстрации в одном рисунке
%
%\input{my_folder/tex/fig-spbpu-main-bld-two-in-one} % пример подключения 2х иллюстраций в одном рисунке

\input{my_folder/tex/tab-more-than-one-page} % пример подключения таблицы на несколько страциц


\begin{table} [htbp]% Пример оформления таблицы
	\centering\small
	\caption{Пример представления данных для сквозного примера по ВКР \cite{Peskov2004}}%
	\label{tab:ToyCompare}		
		\begin{tabular}{|l|l|l|l|l|l|}
			\hline
			$G$&$m_1$&$m_2$&$m_3$&$m_4$&$K$\\
			\hline
			$g_1$&0&1&1&0&1\\ \hline
			$g_2$&1&2&0&1&1\\ \hline
			$g_3$&0&1&0&1&1\\ \hline
			$g_4$&1&2&1&0&2\\ \hline
			$g_5$&1&1&0&1&2\\ \hline
			$g_6$&1&1&1&2&2\\ \hline		
		\end{tabular}
%	\caption*{\raggedright\hspace*{2.5em} Составлено (или/и рассчитано) по \cite{Peskov2004}} %Если проведена авторская обработка или расчеты по какому-либо источнику	
	\normalsize% возвращаем шрифт к нормальному
\end{table}



%% please, before using, read the author guide carefully

\input{my_folder/tex/tab-toy-context-minipage} % пример подключения minipage

\input{my_folder/tex/fig-spbpu-new-bld-autumn-minipage} % пример подключения minipage




\input{my_folder/tex/rules-theorem-like-expressions} 

По аналогии с нумерацией формул, рисунков и таблиц нумеруются и иные текстово-графические объекты, то есть включаем в нумерацию номер главы, например: теорема 3.1. для первой теоремы третьей главы монографии. Команды \LaTeX{} выставляют нумерацию и форматирование автоматически. Полный перечень команд для подготовки текстово-графических и иных объектов находится в подробных методических рекомендациях \cite{spbpu-bci-template-author-guide}. 


\input{my_folder/tex/rules-list-of-environments} % список некоторых окружений


\input{my_folder/tex/theorem-example} %пример оформления теоремы


\input{my_folder/tex/definition-example} %пример оформления определения


Вместо теоремо-подобных окружений для вставки небольших текстово-графических объектов иногда используются команды. Типичным примером такого подхода является команда \verb|\footnote{text}|\footnote{Внимание! Команда вставляется непосредственно после слова, куда вставляется сноска (без пробела). Лишние пробелы также не указываются внутри команды перед и после фигурных скобок.}, где в аргументе \verb|text| указывают текст \textit{подстрочной ссылки (сноски)}.В них \textit{нельзя добавлять веб-ссылки или цитировать литературу}. Для этих целей используется список литературы. Нумерация сносок сквозная по ВКР без точки на конце выставляется в шаблоне автоматически, однако в каждом приложении к ВКР нумерация, зависящая от номера приложения, выставляется префикс <<П>>, например <<П1.1>> --- первая сноска первого приложения. 




%\FloatBarrier % заставить рисунки и другие подвижные (float) элементы остановиться


\section{Выводы} \label{ch2:conclusion}

Текст заключения ко второй главе. Пример ссылок \cite{Article,Book,Booklet,Conference,Inbook,Incollection,Manual,Mastersthesis,Misc,Phdthesis,Proceedings,Techreport,Unpublished,badiou:briefings}, а также ссылок с указанием страниц, на котором отображены те или иные текстово-графические объекты  \cite[с.~96]{Naidenova2017} или в виде мультицитаты на несколько источников \cites[с.~96]{Naidenova2017}[с.~46]{Ganter1999}. Часть библиографических записей носит иллюстративный характер и не имеет отношения к реальной литературе. 

Короткое имя каждого библиографического источника содержится в специальном файле \verb|my_biblio.bib|, расположенном в папке \verb|my_folder|. Там же находятся исходные данные, которые с помощью программы \texttt{Biber} и стилевого файла \texttt{Biblatex-GOST} \cite{ctan-biblatex-gost} приведены в списке использованных источников согласно ГОСТ 7.0.5-2008.
Многообразные реальные примеры исходных библиографических данных можно посмотреть по ссылке \cite{ctan-biblatex-gost-examples}.

Как правило, ВКР должна состоять из четырех глав. Оставшиеся главы можно создать по образцу первых двух и подключить с помощью команды \verb|\input| к исходному коду ВКР. Далее в приложении \ref{appendix-MikTeX-TexStudio} приведены краткие инструкции запуска исходного кода ВКР \cite{latex-miktex,latex-texstudio}.

В приложении \ref{appendix-extra-examples} приведено подключение некоторых текстово-графических объектов. Они оформляются по приведенным ранее правилам. В качестве номера структурного элемента вместо номера главы используется <<П>> с номером главы. Текстово-графические объекты из приложений не учитываются в реферате.



%% Вспомогательные команды - Additional commands
%
%\newpage % принудительное начало с новой страницы, использовать только в конце раздела
%\clearpage % осуществляется пакетом <<placeins>> в пределах секций
%\newpage\leavevmode\thispagestyle{empty}\newpage % 100 % начало новой страницы