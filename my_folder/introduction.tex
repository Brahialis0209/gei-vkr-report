\chapter*{Введение} % * не проставляет номер
\addcontentsline{toc}{chapter}{Введение} % вносим в содержание

Для исследования генной регуляции требуется получать количественные данные по экспрессии генов с учетом пространственной локализации.

Рассмотрим следующую задачу для решения которой изучается генная регуляция. Итак, сначала по изображениям мозга плодовой мушки измеряют уровень экспресси генов, то есть получают количественные данные с учетом пространственной локализации генов. Далее полученные данные используют для изучения поведения мух  в период спаривания - сравнивают уровни экспрессии генов в разных частях мозга у мушек разных полов с их поведением в период спаривания. Под поведением можно понимать их привлекательность друг другу, желание спариваться и др. Далее эти статистические связи от модельных объектов(мушек) можно попробовать распространить на более сложные органимы(мыши, собаки и др.)

В данной работе изучается получение количественных  данных которые можно было бы использовать для приведенной задачи выше. Для выделения на экспериментальных изображениях комплексов молекул РНК будут использованы методы обработки изображений.

Целью данной работы является разработка алгоритма для выделения на экспериментальных изображениях комплексов молекул РНК и применение для анализа паттернов экспрессии генов в мозге плодовой мушки.

Для достижения поставленной цели требуется решить следующие задачи:

\begin{enumerate}[1.]
	\item Изучить методы разделения каналов в экспериментальных биологических изображениях и подобрать пригодные для тестирования в имеющихся данных. Проверить работу методов на тестовых данных из соответствующих статей.
	\item Модифицировать и запрограммировать отобранные методы для процедуры обработки имеющихся данных по экспрессии генов в мозге плодовой мушки, выделить настроечные параметры.
	\item Получить количественные данные по экспрессии генов в мозге плодовой мушки по имеющимся изображениям.
	\item Проанализировать различия в экспрессии генов для разных условий.
\end{enumerate} 

%Целью первой главы, как правило, является всесторонний анализ предмета и объекта исследования, второй --- разработка предложений (алгоритмов, технологий и т.п.) по улучшению какого-либо процесса, протекающих с участием предмета и объекта исследования, третьей --- практическая реализация (имплементация) --- предложений (алгоритмов, технологий и т.п.) в виде программного (или иного) продукта, четвертой --- апробация разработанных в работе предложений и выводы целесообразности их дальнейшей разработки (использованию). 



%% Вспомогательные команды - Additional commands
%\newpage % принудительное начало с новой страницы, использовать только в конце раздела
%\clearpage % осуществляется пакетом <<placeins>> в пределах секций
%\newpage\leavevmode\thispagestyle{empty}\newpage % 100 % начало новой строки