%%% Не мянять - Do not modify
%%
%%
\clearpage                                  % В том числе гарантирует, что список литературы в оглавлении будет с правильным номером страницы
%\hypersetup{ urlcolor=black }               % Ссылки делаем чёрными
%\providecommand*{\BibDash}{}                % В стилях ugost2008 отключаем использование тире как разделителя 
\urlstyle{rm}                               % ссылки URL обычным шрифтом
\ifdefmacro{\microtypesetup}{\microtypesetup{protrusion=false}}{} % не рекомендуется применять пакет микротипографики к автоматически генерируемому списку литературы
%\newcommand{\fullbibtitle}{Список литературы} % (ГОСТ Р 7.0.11-2011, 4)
%\insertbibliofull  
%\noindent
%\begin{group}
\chapter*{Список использованных источников}	
\label{references}
\addcontentsline{toc}{chapter}{Список использованных источников}	% в оглавление 
\printbibliography[env=SSTfirst]                         % Подключаем Bib-базы
%\ifdefmacro{\microtypesetup}{\microtypesetup{protrusion=true}}{}
%\urlstyle{tt}                               % возвращаем установки шрифта ссылок URL
%\hypersetup{ urlcolor={urlcolor} }          % Восстанавливаем цвет ссылок



%\urlstyle{rm}                               % ссылки URL обычным шрифтом
%\ifdefmacro{\microtypesetup}{\microtypesetup{protrusion=false}}{} % не рекомендуется применять пакет микротипографики к автоматически генерируемому списку литературы
%\insertbibliofull                           % Подключаем Bib-базы
%\ifdefmacro{\microtypesetup}{\microtypesetup{protrusion=true}}{}
%\urlstyle{tt}                               % возвращаем установки шрифта ссылок URL








































%%%% Не мянять - Do not modify
%%%
%%%
%\clearpage                                  % В том числе гарантирует, что список литературы в оглавлении будет с правильным номером страницы
%%\hypersetup{ urlcolor=black }               % Ссылки делаем чёрными
%%\providecommand*{\BibDash}{}                % В стилях ugost2008 отключаем использование тире как разделителя 
%%\urlstyle{rm}                               % ссылки URL обычным шрифтом
%%\ifdefmacro{\microtypesetup}{\microtypesetup{protrusion=false}}{} % не рекомендуется применять пакет микротипографики к автоматически генерируемому списку литературы
%%\newcommand{\fullbibtitle}{Список литературы} % (ГОСТ Р 7.0.11-2011, 4)
%%\insertbibliofull  
%\noindent
%%\begin{group}
%%\chapter*{Литература}	
%\label{references}
%%\addcontentsline{toc}{chapter}{Литература}	% в оглавление 
%\begin{thebibliography}{3}
%	% литра
%	\bibitem{prostack} Kozlov, K.N., Baumann, P., Waldmann, J. et al. TeraPro, a system for pro-cessing large biomedical images. Pattern Recognit. Image Anal. 23, 488–497 (2013). https://doi.org/10.1134/S105466181304007X. 
%	\bibitem{cv2} Bradski, G. (2000). The OpenCV Library. Dr. Dobb's Journal of Software Tools.
%	\bibitem{find-contours} http://masters.donntu.org/2014/fknt/metelytsia/library/article11.htm Алгоритмы выделения контуров для сегментации изображений
%	\bibitem{morf} https://habr.com/ru/post/113626/ Математическая морфология
%	\bibitem{find-contours-habr}https://habr.com/ru/post/128753/ Методы нахождения границ изображения
%	\bibitem {pr} https://sourceforge.net/projects/prostack/ Пакет ProStack
%	\bibitem{kk} K.Kozlov Disertation 2012
%	\bibitem{autotools} https://ru.wikipedia.org/wiki/Autotools Система Autotools
%	\bibitem{pacman} https://ru.wikipedia.org/wiki/Pacman Система Pacman
%	\bibitem{cmake} https://ru.wikipedia.org/wiki/CMake Система сборки Cmake
%	
%\end{thebibliography}
%%\printbibliography[env=SSTfirst]                         % Подключаем Bib-базы
%%\ifdefmacro{\microtypesetup}{\microtypesetup{protrusion=true}}{}
%%\urlstyle{tt}                               % возвращаем установки шрифта ссылок URL
%%\hypersetup{ urlcolor={urlcolor} }          % Восстанавливаем цвет ссылок
%
%
%
%%\urlstyle{rm}                               % ссылки URL обычным шрифтом
%%\ifdefmacro{\microtypesetup}{\microtypesetup{protrusion=false}}{} % не рекомендуется применять пакет микротипографики к автоматически генерируемому списку литературы
%%\insertbibliofull                           % Подключаем Bib-базы
%%\ifdefmacro{\microtypesetup}{\microtypesetup{protrusion=true}}{}
%%\urlstyle{tt}                               % возвращаем установки шрифта ссылок URL